\chapter{Giới thiệu}
% \section{Ví dụ}

% \noindent Đây là một ví dụ, sau này sẽ được update nội dung sau.

% If this chapter/section has a star, it won't be in the table of content.
% \section*{Đây là section k dc thêm vào mục lục}
\section{Giới thiệu đề tài}
% \noindent Trong thời đại công nghệ số và đại dịch, việc mua sắm trực tuyến đang trở nên phổ biến hơn bao giờ hết. Điều này đặt ra một thách thức lớn cho các doanh nghiệp, đòi hỏi họ phải sở hữu kênh bán hàng trực tuyến để đáp ứng nhu cầu của khách hàng. Đề tài này nhằm xây dựng một hệ thống thương mại điện tử linh hoạt, ổn định và có tính mở rộng trên nhiều nền tảng khác nhau. Hệ thống này cũng phải dễ dàng bảo trì và nâng cấp trong tương lai. Do đó, việc một doanh nghiệp sở hữu kênh bán hàng trên nền tảng số là vô cùng cần thiết. Mục tiêu của đề tài là xây dựng một hệ thống thương mại điện tử có tính sẵn sàng, tính mở rộng cao, có thể được đưa lên nhiều nền tảng khác nhau một cách dễ dàng, dễ bảo trì, nâng cấp trong tương lai.

\noindent Trong thời đại kỹ thuật số, thương mại điện tử đã trở thành một lực lượng chi phối, cách mạng hóa cách thức hoạt động của các doanh nghiệp và cách thức mua sắm của người tiêu dùng. 
Sự phát triển của các nền tảng mua sắm trực tuyến đã nhấn mạnh sự cần thiết của các hệ thống thương mại điện tử vững chắc, có tính mở rộng và tính sẵn sàng cao. 
Khi ngày càng nhiều người tiêu dùng chuyển sang mua sắm trực tuyến, nhu cầu về các hệ thống thương mại điện tử có thể xử lý khối lượng lớn giao dịch, đảm bảo thời gian ngừng hoạt động tối thiểu và cung cấp trải nghiệm người dùng liền mạch chưa bao giờ cao như hiện nay. Đề tài này nhằm khám phá và phát triển một hệ thống thương mại điện tử đáp ứng các yêu cầu quan trọng này, tập trung vào  các mục tiêu cốt lõi là tính mỏ rộng và tính sẵn sàng cao.\\[0.5cm]
Sự xuất hiện của điện toán đám mây, kiến trúc microservices và các công nghệ cơ sở dữ liệu tiên tiến đã cung cấp cơ hội mới cho việc xây dựng các hệ thống có tính mỏ rộng và tính sẵn sàng cao. 
Tuy nhiên, phát triển một nền tảng thương mại điện tử tận dụng các công nghệ này một cách hiệu quả trong khi duy trì hiệu suất tối ưu, bảo mật và chi phí hiệu quả đặt ra nhiều thách thức lớn. Đề tài này giải quyết các thách thức này bằng cách đề xuất giải pháp cho việc thiết kế và triển khai một hệ thống thương mại điện tử bán các sản phẩm công nghệ phù hợp với nhu cầu của các doanh nghiệp và người tiêu dùng hiện đại.\\[0.5cm]
Tính sẵn sàng cao trong các hệ thống thương mại điện tử đảm bảo rằng nền tảng luôn hoạt động và có thể truy cập được vào mọi lúc, ngay cả khi gặp sự cố phần cứng, lỗi phần mềm hoặc lưu lượng truy cập đột ngột. Sự đáng tin cậy này rất quan trọng để duy trì niềm tin của người tiêu dùng và đảm bảo dòng doanh thu liên tục. 
Tính mở rộng, mặt khác, đề cập đến khả năng của hệ thống xử lý tải tăng lên bằng cách thêm tài nguyên mà không làm giảm hiệu suất. Một nền tảng thương mại điện tử có tính mở rộng có thể cùng doanh nghiệp phát triển để đáp ứng nhiều người dùng, giao dịch và dữ liệu hơn khi cần.\\[0.5cm]
Bằng cách phát triển một hệ thống thương mại điện tử với sự tập trung mạnh mẽ vào tính mở rộng và tính sẵn sàng cao, đề tài này sẽ đóng góp vào việc xây dựng các hệ thống thương mại điện tử, cung cấp một lộ trình cho các doanh nghiệp muốn nâng cao sự hiện diện trực tuyến và hiệu quả hoạt động của họ. 
Các phát hiện và khuyến nghị được trình bày trong luận văn không chỉ mang lại lợi ích cho các nhà phát triển, kỹ sư mà còn cung cấp thông tin để các nhà lãnh đạo doanh nghiệp đưa ra các quyết định chiến lược khi đầu tư vào các giải pháp thương mại điện tử cho doanh nghiệp mình.


\section{Mục tiêu và phạm vi của đề tài}
\noindent Hệ thống bao gồm những chức năng chính cho một trang web thương mại điện tử bán sản phẩm công nghệ và một số tính năng, đặc điểm nổi bật như tính sẵn sàng cao, tính mở rộng cao, đáp ứng được lưu lượng truy cập biến động của người dùng.

\section{Cấu trúc đồ án}
\noindent Nội dung của đồ án được trình bày bao gồm những chương sau

\begin{itemize}
    \item \textbf{\underbar{Chương 1:}} Giới thiệu
    \item \textbf{\underbar{Chương 2:}} Cơ sở lý thuyết
    \item \textbf{\underbar{Chương 3:}} Phân tích yêu cầu
    \item \textbf{\underbar{Chương 4:}} Thiết kế hệ thống
    \item \textbf{\underbar{Chương 5:}} Hiện thực hệ thống
    \item \textbf{\underbar{Chương 6:}} Triển khai, kiểm thử và đánh giá hệ thống
    \item \textbf{\underbar{Chương 7:}} Tổng kết
\end{itemize}